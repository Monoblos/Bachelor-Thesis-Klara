%!TEX root = ../thesis.tex

\chapter{Einleitung}

Für fundiertes Verst"andnis von Java und um ein Grundlagen-orientiertes lernen zu erm"oglichen hilft es anf"anglich nur mit einem normalen Texteditor Code zu schreiben und diesen selbst zum compilieren. Da gerade Anf"angern das erkennen und finden von Fehlern schwer f"allt ist allerdings auch ein Tool um dies zu vereinfachen \"au\ss{}erst hilfreich.

\section{Problemstellung} 

Im Java Umfeld gibt es eine Reihe von Tools zum debuggen von Anwendungen, jedoch sind diese alles große und meist komplexe Programme. Zudem sind diese auch meist Teil einer Programmierumgebung und nicht ohne diese einsetzbar. 

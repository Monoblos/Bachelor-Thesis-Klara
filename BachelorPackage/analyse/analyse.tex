%!TEX root = ../thesis.tex

\chapter{Analyse}

Diese Kapitel beschäftigt sich mit der Analyse, was genau Tracing ist und welche Anwendungen es bereits in diesem Gebiet gibt. Außerdem werden Anforderungen dargelegt und durch Ideen und Szenarien weiter erläutert.

\section{Tracing} 

Tracing beschreibt eine spezielle Form des loggings, bei der Daten gesammelt werden um den Ablauf eines Programms rekonstruieren zu können. Dies ermöglicht, im Fehlerfall, Entwicklern nachvollziehen zu können, wo und wieso welches Problem aufgetreten ist. Die erzeugten Daten beim tracing werden als Programmtrace, oder kurz Trace bezeichnet, was sich ins deutsche als "`Spur"' übersetzen lässt. Tracing stellt die umfassendste Art des Loggings dar, da bis ins kleinste Detail die Ausführung protokolliert wird. Aus diesem Grund ist sie im produktiven Betrieb auch ungeeignet, da Massen an Daten erzeugt werden.

\section{Ähnliche Anwendungen} 



\section{Anforderungen} 
\label{sec:anforderungen}



\section{Idee} 



\section{Szenario} 


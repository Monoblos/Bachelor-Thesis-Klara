%!TEX root = ../thesis.tex

\chapter{Fazit}

Im Rahmen der Bachelorarbeit wurden die geforderten Funktionen erfolgreich umgesetzt. Das entstandene Programm bietet grundlegende Möglichkeiten zur Fehlersuche in Java-Programmen mit einer Kommandozeilen Oberfläche. Diese ist sehr simpel gestaltet um auch für Programmieranfänger intuitiv bedienbar zu sein.

Der höchste Aufwand der Arbeit lag bei der Entscheidung für eine Technologie, mit welcher die Aufgabe umgesetzt werden konnten. Hierbei musste ich mich in einige verschiedener Technologien einarbeiten und jeweils kleine Programme schreiben, um deren Nutzbarkeit zu evaluieren. Aufgrund der Komplexität von den meisten Frameworks gestaltete sich dies als sehr zeitintensive Aufgabe, weil bei Problemen immer darauf geachtet werden musste, ob die Ursache bei Implementierung oder Technologie liegt. Dies hat jedoch auch einige interessante Einblicke gegeben, da alle getesteten Frameworks durchaus sinnvolle Einsatzgebiete haben.

Bei der Umsetzung lagen die Schwierigkeiten beim finden eines Formats in der Optionen weitergegeben werden, sowie der Struktur um Transformatoren modular ohne Code duplikation implementieren zu können. Für die Optionen wäre bei Erweiterungen eine Vereinfachung mit Hilfe von kapselnden Klassen vorstellbar, welche Beispielsweise alle Filteroptionen kombiniert. Bei den Transformatoren habe ich viel Aufwand darauf verschwendet, eine Möglichkeit zu finden diese in die bestehende Architektur von ASM zu integrieren. Viel des dafür geschriebenen Codes zeigte sich später als nutzlos, nachdem die Struktur umgebaut wurde.

Im allgemeinen war das Thema eine persönliche Bereicherung, da es viele neue Einblicke in die Programmiersprache gegeben hat. Zwar war mir die Existenz des Bytecodes bewusst, die spezifische Umwandlung von Java-Code zu Bytecode-Befehlen und die Möglichkeit diesen direkt zur Laufzeit zu manipulieren waren jedoch gänzlich neu. Mit aktiv weiterentwickelten Projekten wie ASM sehe ich Potenzial für viele verschiedene Anwendungen in diesem Umfeld.

\section{Erweiterungsmöglichkeiten} 

Umbau der gewrappten Variablen das diese tiefe "Anderungen tracken k"onnen

Ausgabe des Branchings anstelle Zeilen (-b)

Invisible Asserts (Fehler-case definieren und passend konfigurieren)

Auto-Detect of relevant classes (Not just the class containing main)

Support for jars?

Men"u zum konfigurieren, "ahnlich interactive, siehe \ref{sec:console}

Speichern von Konfigurationen erstellt mit Men"u zum einfachen wiederholen von Debug-vorg"angen.

Parameterabstraktion die f"ur alle UI's benutzbar ist (Buchstabe, Frage, Menüposition und Label, Wert-Validierung)

Ant-Plugin zum Aufruf

\section{Weitere Anwendungsbereiche} 

Heavy Obfuscation after compile (goto and stuff)

Invisible Code generators

"`Makro"' calls (Replace "`method"' class with specific code)

CTF Aufgabe

Traceable generieren für Server
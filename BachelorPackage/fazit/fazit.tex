%!TEX root = ../thesis.tex

\chapter{Fazit}

Im Rahmen der Bachelorarbeit wurden die geforderten Funktionen erfolgreich umgesetzt. Das entstandene Programm bietet grundlegende Möglichkeiten zur Fehlersuche in Java-Programmen mit einer Kommandozeilen-Oberfläche. Diese ist sehr simpel gestaltet, um auch für Programmieranfänger intuitiv bedienbar zu sein.

Der höchste Aufwand der Arbeit lag bei der Entscheidung für eine Technologie, mit welcher die Aufgaben umgesetzt werden konnten. Hierbei musste ich mich in einige verschiedene Technologien einarbeiten und jeweils kleine Programme schreiben, um deren Nutzbarkeit zu evaluieren. Aufgrund der Komplexität der meisten Frameworks gestaltete sich dies als sehr zeitintensive Aufgabe, weil bei Problemen immer darauf geachtet werden musste, ob die Ursache bei der Implementierung oder der Technologie liegt. Dies hat jedoch auch einige interessante Einblicke gegeben, da alle getesteten Frameworks durchaus sinnvolle Einsatzgebiete haben.

Bei der Umsetzung lagen die größten Herausforderungen beim Finden eines Formats, in dem Optionen weitergegeben werden, sowie der Struktur, um Transformatoren modular ohne Code-Duplikation implementieren zu können. Für die Optionen wäre bei Erweiterungen eine Vereinfachung mit Hilfe von kapselnden Klassen vorstellbar, welche beispielsweise alle Filteroptionen kombiniert. Bei den Transformatoren habe ich viel Aufwand darauf verschwendet, eine Möglichkeit zu finden, diese in die bestehende Architektur von ASM zu integrieren. Viel des dafür geschriebenen Codes zeigte sich später als nutzlos, nachdem die Struktur umgebaut wurde.

Im Allgemeinen war das Thema eine persönliche Bereicherung, da es viele neue Einblicke in die Programmiersprache gegeben hat. Zwar war mir die Existenz des Bytecodes bewusst, die spezifische Umwandlung von Java-Code zu Bytecode-Befehlen und die Möglichkeit, diese direkt zur Laufzeit zu manipulieren waren jedoch gänzlich neu. Mit aktiv weiterentwickelten Projekten wie ASM sehe ich Potenzial für viele verschiedene Anwendungen in diesem Umfeld.

\section{Erweiterungsmöglichkeiten} 

Mit der grundlegenden Funktionalität gegeben, sehe ich Potenzial für verschiedene Erweiterungen.
Wie in Kapitel~\ref{sec:console} bereits beschrieben, wäre eine Menü-Oberfläche für komplexere Konfigurationen denkbar. Somit könnte der Anwender in beliebiger Reihenfolge die Optionen wählen, bis ihm die Konfiguration gefällt. Die Menüpunkte selbst könnten dann Werte ähnlich erfragen wie es beim interaktiven Modus implementiert ist.
Mit der Möglichkeit, aufwändige Konfigurationen in einem Menü zu erstellen, sollte auch eine Option zum Speichern dieser Konfiguration implementiert werden. Ein besonders einfaches Format hierfür wäre die Parameterliste, wie sie auch direkt übergeben werden kann. Wenn diese zu groß wird oder weitere Optionen hinzukommen, die nur über das Menü verfügbar sind, sollte eine XML oder JSON-Struktur verwendet werden.
Da das Menü eine 3. Oberfläche darstellen würde, welche die gleichen Optionen bieten soll, wäre eine Abstraktion der Parameter sinnvoll. So könnten alle Informationen zu einer Option in einem Format gebündelt werden, das alle Oberflächen benutzen können. Somit sollten alle Informationen, von dem Argumentname als Übergabeparameter bis hin zu Wert-Validierung, enthalten sein. Dies würde sicherstellen, dass die gleiche Funktionalität über alle Oberflächen erreichbar ist und Code-Duplikationen vermeiden.

Eine zusätzlich denkbare Option ist ein gröberes Branching zu untersuchen, als Alternative zur Ausgabe jeder Zeile. Ausgegeben würde in diesem Fall beispielsweise nur, dass eine Schleife 17 Durchläufe hatte oder ein Switch-case zu Fall 3 gesprungen ist.
Dies würde bei größeren Abläufen die Übersichtlichkeit erhöhen, besonders wenn lange laufende Schleifen vorhanden sind.

Weitere kleinere mögliche Verbesserungen sind ein Ant-Plugin und Support für das debuggen von jar-Archiven. Ein Ant-Plugin könnte als Integration in einer beliebigen Entwicklungsumgebung genutzt werden. Jar-Archive können momentan noch nicht untersucht werden, lediglich .class-Dateien.
\todo{Check jar support}

Desweiteren habe ich Ideen für fünf große Erweiterungen:

Unsichtbare Überprüfungen, welche in einer Konfiguration definiert werden. Diese funktionieren ähnlich wie eine "`assert"'-Anweisung, sind aber nicht Teil des Codes. Dies ist syntaktisch besser, da sie lediglich Tests darstellen und nicht zum Programm gehören. Somit bleibt der Code übersichtlicher und die Ausführung wird nur dann verlangsamt, wenn die Überprüfungen auch ausgeführt werden sollen.

Automatisches Erkennen der relevanten Klassen. Wenn eine Klasse \code{some.package.Foo} untersucht werden soll, gilt selbiges wahrscheinlich für andere Klassen im Paket \code{some.package}. Hier könnte eine Reihe von logischen Abhängigkeiten gebildet werden, um das manuelle Setzen von Filtern nahezu überflüssig zu machen.

Ähnlich dazu könnten verwendete Typen modifiziert werden, um Änderungen an den gekapselten Variablen zu erkennen. Hierbei könnte auch eine Prüfung des Stacktrace eingebaut werden, welche nur in bestimmten Fällen Ausgaben erzeugt. Damit wäre auch erkennbar, wenn ein gespeicherter Punkt nicht ersetzt wird, sondern lediglich die Koordinaten sich ändern.

Erzeugen eines maschinell lesbaren Trace, anstelle eines menschlich lesbaren. Dieser könnte viel performanter erzeugt werden als es beispielsweise bei Python Tutor der Fall ist, da keine Pausierung der \ac{VM} stattfindet. Ein anderes Programm könnte diesen dann in einem für Entwickler geeigneten Format visualisieren.

Logging mit bekannten Frameworks wie Log4j erzeugen und die Ergebnisse speichern. Solche veränderten Klassen könnten auf Server zum Einsatz kommen, um bei Problemen Logging mittels des Frameworks zu aktivieren ohne etwas am Code oder Server-Setup ändern zu müssen. Da Server auf Grund der Anforderungen an Antwortzeiten und häufiger Komplexität der Umgebung nicht einfach zu debuggen sind, wäre dies eine schnellere Alternative.

\section{Weitere Anwendungsbereiche} 

Zusätzlich zu möglichen Erweiterungen der bestehenden Anwendung sehe ich auch Potenzial für andere Anwendungsgebiete der Technologie. Bei allen ausgeführten Änderungen am Bytecode wurde bisher kein Konstrukt erzeugt, was nicht auch durch Änderung des Sourcecodes möglich gewesen wäre. Hier bieten sich jedoch neue Möglichkeiten.

Obfuscation beschreibt das Verändern des Codes in solcher Weise, dass der originale Code nur noch sehr schwer oder in keiner Weise wiederherstellbar ist. Dies kommt besonders bei kommerzieller Software zum Einsatz, wenn das Kompilat der Sprache Rückschlüsse auf den Code geben kann. Dies ist unter anderem bei allen Sprachen die Bytecode verwenden der Fall.
Hier könnte mit ASM das übersetzte Programm stark umgebaut werden, ohne die eigentliche Funktionalität zu ändern. Mit Hilfe von Sprüngen quer durch den Code und den potenziellen kleinen Veränderungen der Aufruf-Struktur sollten herkömmliche Decompiler nicht mehr in der Lage sein, einen sinnvollen Code daraus zu erzeugen. Allerdings muss dabei stark darauf geachtet werden, die Funktionalität wirklich nicht einzuschränken und der Speicherverbrauch sowie die Laufzeit ändern sich wahrscheinlich.

Eine der Veränderungen, die hierbei durchgeführt werden können, aber die Laufzeit potenziell verbessern, sind "`Makros"'. Dies könnte als Annotation an statischen Methoden realisiert werden. Aufrufe dieser werden durch den Inhalt der Methode ersetzt, um den Kontextwechsel zu vermeiden und potenziell das Laden der Utility-Klasse gänzlich unnötig zu machen.

Genau gegensätzlich dazu könnte ein unsichtbarer Code erstellt werden. Dies wären Klassen mit leeren Methoden, deren Code beim Laden generiert wird. In der Verwendung wäre ein Unterschied nicht erkennbar, jedoch bietet er eine denkbare Alternative zu klassischen Code-Generatoren.

Alle diese Mittel, welche einen schwer nachvollziehbaren Code erzeugen, können auch bei der Erstellung von Aufgaben für Wettbewerbe verwendet werden. Hier könnte eine Reversing-Aufgabe, welche sonst meist nur aus Maschinencode besteht, in Form von komplexem Bytecode erfolgen, welcher einen weiteren Code generiert, der das eigentliche Programm darstellt. So könnte eine große Verschachtelung erstellt werden, welche eine Herausforderung zum Entschlüsseln darstellt.

\chapter{Verfügbarkeit der Anwendung}

Der gesamte Code der Anwendung ist online unter diesem Link verfügbar:

https://github.com/Monoblos/github/labels/2013-new-repo-view

Die URL ist auch in gekürzte Version, sowie als eine QR-Code Repräsentation verfügbar:

https://goo.gl/QiQEqI

QR-Code

Ebenfalls sind alle Dateien zum erstellen dieser schriftlichen Ausarbeitung online verfügbar:

https://github.com/Monoblos/Bachelor-Thesis-Klara/tree/V1

https://goo.gl/gJSoTj

QR-Code